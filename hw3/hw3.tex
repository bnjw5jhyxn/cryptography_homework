\centerline{Steve Hsu\hfill homework 3}
\item{1.}

$2$ is a primitive root of $19$.
$$\eqalign{
2 &\not\equiv 1 \pmod{19}\cr
2^2 = 4&\not\equiv 1 \pmod{19}\cr
2^3 = 8 &\not\equiv 1 \pmod{19}\cr
2^4 = 16 &\not\equiv 1 \pmod{19}\cr
2^5 = 32 \equiv 13 &\not\equiv 1 \pmod{19}\cr
2^6 = 26 \equiv 7 &\not\equiv 1 \pmod{19}\cr
2^7 = 14 &\not\equiv 1 \pmod{19}\cr
2^8 = 28 \equiv 9 &\not\equiv 1 \pmod{19}\cr
2^9 = 18 &\not\equiv 1 \pmod{19}\cr
}$$
Since the order of $2$ modulo $19$ must divide $18$,
and the order is greater than $9$, the order of $2$ must be $18$
and $2$ is therefore a primitive root of $19$.
\medskip
$7$ is not a primitive root of $19$
because $7^3 \equiv (2^6)^3 \equiv 2^18 \equiv 1 \pmod{19}$,
so $7$ is not a primitive root of $19$.
\bigskip
\item{2.}

Assume that $a^2 \equiv 1 \pmod p$.
By definition of congruence, $p \vert (a^2 - 1) = (a + 1)(a - 1)$.
Since $p$ is prime and $p \vert (a+1)(a-1)$,
$p \vert (a+1)$ or $p \vert (a-1)$.
By definition of congruence, $a \equiv -1 \pmod p$ or $a \equiv 1 \pmod p$, as desired.

$3^2 = 9 \equiv 1 \pmod 8$, but $3 \not\equiv 1 \pmod 8$ and $3 \not\equiv -1 \pmod 8$.
\bigskip
\item{3.}

Notice that $g^a \equiv h \equiv g^b \pmod p$.
Assume without loss of generality that $a \ge b$.
Multiplying both sides by $(g^{-1})^b$, we have that
$g^{a-b} \equiv 1 \pmod p$.
Since $g$ is a primitive root of $p$,
the order of $g$ modulo $p$ must be $p - 1$,
so $(p - 1) | (a - b)$, as desired.
\bigskip
\item{4.}

Let $a = \log _g (h_1)$ and $b = \log _g (h_2)$.
Notice that $g^{a+b} \equiv g^a g^b \equiv h_1 h_2 \pmod p$.
Therefore, $a + b \equiv \log _g (h_1 h_2) \pmod p$.
Suppose $c$ is an integer such that $g^c \equiv h_1 h_2 \pmod p$.
Then $g^c \equiv g^{a+b} \pmod p$.
Multiplying both sides by $(g^{-1})^c$, we have that
$g^{a+b-c} \equiv 1 \pmod p$.
Since $g$ is a primitive root of $p$,
$a + b - c \equiv 0 \pmod{p-1}$.
Adding $c$ to both sides, we have that
$a + b \equiv c \pmod{p-1}$, as desired.
\bigskip
\item{5.}
\itemitem{a.} false
\itemitem{b.} true
\itemitem{c.} false
\itemitem{d.} false
\itemitem{e.} true
\itemitem{f.} true
\itemitem{g.} false
\itemitem{h.} true
\itemitem{i.} false
\itemitem{j.} true
\bye
