\centerline{Steve Hsu\hfill homework 4}
\item{1.}

Let $r_1 = \max \{n, p\}$ and $r_2 = \min \{n, p\}$.
Given $r_k$ and $r_{k+1}$, we want to find $q_{k+2}$ and $r_{k+2}$ such that
$r_{k+2} < r_{k+1}$ and $r_k = q_{k+2} r_{k+1} + r_{k+2}$.
Notice that if $r_{k+1} \le {1 \over 2} r_k$, then since $r_{k+2} < r_{k+1}$,
we have that $r_{k+2} < {1 \over 2} r_k$.
If $r_{k+1} > {1 \over 2} r_k$, then since $r_{k+2} < r_k - r_{k+1}$,
we have that $r_{k+2} < {1 \over 2} r_k$.
Therefore, it takes at most $2 \log r_1 = 2k$ steps
for Euclid's algorithm to terminate.
At each step, the algorithm performs one division
in order to calculate $q_{k+2}$ and $r_{k+2}$
and performs two multiplications and two additions to write
$r_{k+2}$ as linear combinations of $n$ and $p$ given the expressions of
$r_k$ and $r_{k+1}$ as linear combinations of $n$ and $p$.
Therefore, the algorithm takes $O(2k(2k^2 + 2k)) = O(k^3)$ time in total.
\bigskip
\item{2.}

Alice creates a public key $e$ and a private key $d$
such that $d \equiv e^{-1} \pmod{(p - 1)(q - 1)}$.
Take $e = 4783$ and $d = 2383$.
Bob encrypts the message, $m = 1231$,
by raising it to the power of $e \pmod{pq}$.
In this case, he gets $m^e \equiv 1231^{4783} \equiv 4032 \pmod{pq}$.
Alice decrypts the message by raising the encrypted message
to the power $d \pmod{pq}$,
obtaining $m^{ed} \equiv 4032^{2383} \equiv 1231 \pmod{pq}$
and recovering Bob's message.
\bigskip
\item{3.}

Since $2491 - 1 = 2490 = 2(1245)$,
a Miller-Rabin witness for the compositeness of $2491$
is an integer $a$ such that $\gcd (a, n) = 1$ and
$a^{1245} \not\equiv -1 \pmod{2490}$.
Since $2^{1245} \equiv 2405 \pmod{2490}$,
we have that $2$ is a Miller-Rabin witness for the compositeness of $2491$.
(We don't need to check $2^{2490}$ since if it is $1 \pmod{2491}$,
then we saw a $1$ before a $-1$ and $2491$ is composite.
If it isn't $1 \pmod{2491}$, then $2$ is a Fermat witness for
the compositeness of $2491$.)
\bigskip
\item{4.}

If $\gcd (a, 3) = 1$, then by Fermat's Little Theorem,
we have that $a^{561} \equiv (a^{2})^{280} a \equiv 1^{280} a \equiv a \pmod 3$.
Otherwise, $\gcd (a, 3) = 3$, so $a \equiv 0 \pmod 3$ and
$a^{561} \equiv 0 \equiv a \pmod 3$.

If $\gcd (a, 11) = 1$, then by Fermat's Little Theorem,
we have that $a^{561} \equiv (a^{10})^{56} a \equiv 1^{56} a \equiv a \pmod{11}$.
Otherwise, $\gcd (a, 11) = 11$, so $a \equiv 0 \pmod{11}$ and
$a^{561} \equiv 0 \equiv a \pmod{11}$.

If $\gcd (a, 17) = 1$, then by Fermat's Little Theorem,
we have that $a^{561} \equiv (a^{16})^{35} a \equiv 1^{35} a \equiv a \pmod{17}$.
Otherwise, $\gcd (a, 17) = 17$, so $a \equiv 0 \pmod{17}$ and
$a^{561} \equiv 0 \equiv a \pmod{17}$.

Since $a^{561} \equiv a \pmod 3$, $a^{561} \equiv a \pmod{11}$, and
$a^{561} \equiv a \pmod{17}$,
we have that $3 \vert (a^{561} - a)$, $11 \vert (a^{561} - a)$,
and $17 \vert (a^{561} - a)$.
By midterm question (5a), it follows that
$(3)(11)(17) = 561 \vert (a^{561} - a)$, so
$a^{561} \equiv a \pmod{561}$, as desired.
\bye
