\def\section{\bigskip\noindent\bf}
\def\natural{{\bf N}}
\def\integer{{\bf Z}}
\def\field{{\bf F}}
\def\lcm{\hbox{\rm lcm}}
\def\floor#1{\vert#1\vert}
{\parindent=0pt
Steve Hsu\par
Cryptography\par
Professor Saraf\par
27 April 2018
}

\baselineskip=24pt
\centerline{A Brief Overview of Homomorphic Encryption}
{\section Examples of Homomorphism in Cryptography}

\proclaim Definition. An encryption function $E$ is {\bf homomorphic}
with respect to an operation $\cdot$ if there is an operation $\star$
such that $E(x) \star E(y) = E(x \cdot y)$ for all $x$ and $y$
in the domain of $E$.

It turns out that RSA is homomorphic with respect to multiplication.
Let $p$ and $q$ be large primes and let $n = pq$.
Let $e$ be a public key and let $d = e^{-1} \pmod {(p - 1)(q - 1)}$.
Let $E : m \mapsto m^e \pmod n$ be the encryption function.
Notice that the encryption of the product of two messages is
$E(m_1) E(m_2) = m_1 ^e \pmod p m_2 ^e \pmod p = (m_1 m_2) ^e \pmod p = E(m_1 m_2)$.
Here, $\cdot$ and $\star$ are both multiplication modulo $n$.

As we will see in the coming sections,
homomorphism is an interesting property
that is present in many encryption schemes.

The ElGamal cryptosystem, commonly used to encrypt symmetric keys used in other cryptosystems,
is also homomorphic.
In this system, Alice generates a key pair by first defining a group $G$.
For the rest of this description, we will refer to the group operation as ``multiplication.''
This group must be cyclic (there should be an element $g \in G$
such that $G = \{g^n : n \in \natural\}$).
Let $q$ be the order of $G$.
Alice's private key is a uniformly element $x \in [q - 1]$,
where $[n] = \{1, 2, \ldots, n\}$.
Alice's public key is $(G, q, g, h)$, where $h = g^x$.

In order to encrypt a message $m \in G$, Bob first chooses
a uniformly random element $y \in [q - 1]$,
which he keeps private, and computes $j = g^y$.
The encryption function is then $E_y : m \mapsto (j, m h^y)$.
Notice that $h^y = (g^x)^y = g^{xy}$.
In order to decrypt the message, Alice then applies the decryption function
$D : (j, c) \mapsto c j^{-x}$.
This procedure recovers $m$ because
$c j^{-x} = m g^{xy} j^{-x} = m g^{xy} g^{-yx} = m$.

As in RSA, we can efficiently compute these quantities
efficiently using fast exponentiation.
The security of the ElGamal system relies on an attacker's inability
to compute $g^{xy}$ given $h = g^x$ and $j = g^y$,
which are both transmitted and therefore publicly known.
If $G = \field _p ^\times = [p - 1]$ under multiplication modulo $p$,
then this problem is equivalent to the discrete log problem.

Define $(a,b) \star (c,d)$ to be $(ab, cd)$.
Notice that
$$\eqalign{
E_{y_1} (m_1) \star E_{y_2} (m_2) &= (g^{y_1}, m_1 h^{y_1}) \star (g^{y_2}, m_2 h^{y_2})\cr
&= (g^{y_1} g^{y^2}, m_1 m_2 h^{y_1} h^{y^2})\cr
&= (g^{y_1 + y_2}, m_1 m_2 h^{y_1 + y^2})\cr
&= E_{y_1 + y_2} (m_1 m_2)\cr
}$$
so the ElGamal cryptosystem is homomorphic with respect to multiplication.

{\section Cryptographic Voting}

Imagine that we attempt to conduct an election by sending ballots
through an insecure system to an election authority that then counts them
and publishes the result in a publicly verifiable way
without revealing the contents of the individual ballots.

The Paillier cryptosystem is suitable for this purpose.
Alice produces two large primes $p$ and $q$
with a special property.
Let $n = pq$ and let $\phi (n) = (p - 1)(q - 1)$ be the totient of $n$.
We require that $\gcd (n, \phi (n)) = 1$.
Let $\lambda = \lcm (p - 1, q - 1)$.
Let $g$ be a random element of $\integer _{n^2} ^\times$,
where $\integer _{n^2} ^\times$ is the subset of $[n^2]$
consisting of those numbers with multiplicative inverses modulo $n^2$.
Define the function $L : x \mapsto \floor{(x - 1) / n}$.
We need $g$ to be such that
$\mu = (L(g^\lambda \bmod n^2))^{-1} \pmod n$ exists.
The public key is the pair $(n, g)$ and
the private key is the pair $(\lambda, \mu)$.

In order to encrypt a message $m$,
Bob selects a random element $r \in [n - 1]$.
The encryption function is $E_r : m \mapsto g^m r^n \pmod{n^2}$
and the decryption function is $D : c \mapsto L(c^\lambda \bmod n^2) \mu \pmod n$.
Paillier gives in his paper a detailed proof that $D$ recovers $m$.
This proof establishes a relationship between the function $L$
and the discrete logarithm base $1 + n$ modulo $n^2$.
We are interested in the homomorphism of $E$ with respect to addition.
Notice that
$$\eqalign{
E_{r_1} (m_1) E_{r_2} (m_2) &= g^{m_1} r_1 ^n g^{m_2} r_2 ^n \pmod n^2\cr
&= g^{m_1 + m_2} (r_1 r_2)^n \pmod n^2\cr
&= E_{r_1 r_2} (m_1 + m_2)\cr
}$$

Danny, who gave a lecture to the cryptography class at Rutgers
on May 20, 2018, presented an application of the Paillier system
to secure voting.
In order to run an election, an election authority can create a key pair.
If the ballot has $p$ options, then the bit representation of each vote
can be divided into $p + 2$ segments.
The first segment contains some zeros,
the second segment contains some random bits,
and each of the other $p$ sections contains either $1$ or $0$,
depending on whether the voter voted for option $p$.
The voter selects a value of $r$ at random
and applies $E_r$ to the ballot, then publishes the encrypted ballot.
The public can then multiply all of the encrypted ballots together
to obtain an encryption of the sum of the votes.
The election authority can decrypt this sum and publish the result of the election,
which the public can check to make sure that it encrypts to the encrypted sum.
Notice that the decryption function does not depend on $r$,
so there is no need to compute the product of the voters' values of $r$.
If the validity of a ballot is in question,
then the election authority can decrypt it.
The public can verify that the ballot is invalid if
the voter publishes his or her value of $r$ along with the decrypted ballot.

{\section Yao's Millionaire Problem}

Imagine that two millionaires are bragging about their wealth
and want to find out which of them is wealthier.
Unfortunately, stating their net worth could leak business secrets,
so they want to make the comparison without revealing their exact wealth.
Hsiao-Ying Lin and Wen-Guey Tzeng give a solution that uses homomorphic encryption.

Let $a$ and $b$ be binary representations the wealth of the two people
padded with leading zeroes so that they are the same length.
Given a bit-string $x = x_n x_{n-1} \cdots x_0$, we define the set
$S_x ^1 = \{x_n x_{n-1} \cdots x_{i+1} 1 : x_i = 1\}$ and
$S_x ^0 = \{x_n x_{n-1} \cdots x_{i+1} 1 : x_i = 0\}$.

\proclaim Claim. $a > b$ if and only if $S_a ^1 \cap S_b ^0 \ne \emptyset$.

$(\Rightarrow)$ Assume that $a > b$.
Then at the first point of difference between the binary representations
of $a$ and $b$ is an index $i$ such that $a_i = 1$ and $b_i = 0$.
By definition, $a_n a_{n-1} \cdots a_{i+1} 1 \in S_a ^1$ and
$b_n b_{n-1} \cdots b_{i+1} 1 \in S_b ^0$.
Since index $i$ is the first point of difference,
we have that $a_j = b_j$ for all $j$ such that $j > i$.
Therefore, $a_n \cdots a_{i+1} 1 = b_n \dots b_{i+1} 1$
and the intersection is nonempty.

$(\Leftarrow)$ Assume that $S_a ^1 \cap S_b ^0 \ne \emptyset$.
Take some string $x_n x_{n-1} \cdots x_{i+1} 1$ contained in the intersection.
Since the string is a member of $S_a ^1$, we have that
$x_n \cdots x_{i+1} 1$ is a prefix of $a$ and
$x_n \cdots x_{i+1} 0$ is a prefix of $b$.
Clearly, $a > b$.

We have now reduced the problem to finding out if two sets
have a nonempty intersection without exposing the contents of the sets.
We can do this with ElGamal, which is multiplicatively homomorphic.
Alice creates a key pair, then constructs a $2 \times n$ table $T$
with the following property.
If $a_i = 0$, then $T[0, i] = E_{y_1} (1)$, where $1$ is the identity element of $G$,
for a randomly selected $y_1$ and
$T[1, i] = E_{y_2} (r)$ for randomly selected $y_2$ and $r$.
Notice that in this case, she can simply choose a random pair $(g_1, g_2) \in G^2$.
If $a_i = 1$, then $T[0, i] = E_{y_1} (r)$ and $T[1, i] = E_{y_2} (1)$.

Alice then sends the table to Bob.
For each string $t = t_n t_{n-1} \cdots t_i \in S_b ^0$,
Bob computes $c_t = T[t_n, n] \star T[t_{n-1}, n-1] \star \cdots \star T[t_i, i]$.
Notice that if $t \in S_a ^1$, then $c_t = E_y (1)$ for some $y$ and
if $t \notin S_a ^1$, then $c_t$ is a random pair $(g_1, g_2)$.
Bob then raises each $c_t$ to a random power $k_t$
so that Alice cannot deduce which table entries
were starred together to produce the $c_i$'s.
He then produces enough random pairs so that there is a total of $n$ pairs,
which he then sends to Alice.
Alice then decrypts the $n$ pairs and sees if any of them are $1$.
If there is no $1$, then the intersection of $S_a ^1$ and $S_b ^0$ is empty
and hence $a \le b$.
If there is a $1$, then as long as $G$ is sufficiently large,
there is a high probability that the intersection is nonempty
and hence $a > b$.

{\section Other Desirable Applications}

In 1978, Rivest, Adelman, and Dertouzos conjectured that
a fully homomorphic encryption scheme,
which should be homomorphic with respect to both addition and multiplication,
should exist.
They give some examples of homomorphic schemes with weak security
and describe a potential application of fully homomorphic encryption.

In their example, a bank wants to hire an outside company to manage its database of records.
Since these records are confidential, the bank needs the records to be encrypted.
On the other hand, the bank also wants to be able to make useful queries
to the database without having to download the entire database every time
(otherwise, it might as well keep the database in-house).
Homomorphic encryption of the data would help here
because the bank can request, for example, the sum of all account balances.
The database can then simply perform the corresponding operation on the encrypted data
and send the result back to the bank.
Additionally, the bank also wants to search in the encrypted database.
In this case, the bank expresses its query as a circuit
and sends the circuit, as well as the encryption of the desired output, to the database,
which performs the relevant operations and returns
the desired documents to the bank.

{\section Yao's Garbled Circuit}

Andrew Yao, who described the millionaires' problem above,
also created a system for arbitrary garbled circuit computation.
In this situation, Alice has a function she wants to compute
and Bob has computing resources available.
Alice wants Bob to perform the computation in such a way
that he cannot find unencrypted inputs or outputs of the function.
Yao gives a solution, but unfortunately,
this system requires transmitting information proportional to the size of the circuit.
Since the circuit must be large enough to process all of Alice's data,
Alice can use the same computational power by downloading
her data and performing the computation herself.

His solution does, however, use an interesting procedure for the exchange of secret bits
called oblivious transfer, described by Michael Rabin.
This procedure avoids the problem of someone having to give his or her bit first
by allowing Alice to deduce Bob's secret from the fact that Bob knows Alice's secret.
We must, however, make the assumption that if Alice or Bob uses
the secret bit, the other person will know.
In his example, Rabin imagines that the secret bits are
the passwords for files and that Alice and Bob will be notified if
the other person's file is opened.

Let $s_a$ be the bit in Alice's possession and let $s_b$ be the bit in Bob's possession.
Alice picks two large random primes $p$ and $q$ and sends $n = pq$ to Bob.
Bob then picks a random number $x$ and sends $c \equiv x^2 \pmod n$ to Alice.
Since Alice can factor $n$, she can compute a square root of $c$ modulo $n$.
She sends one such square root $x_1$ to Bob.
If $x_1 \equiv \pm x \pmod n$, then Bob does not gain any information.
Otherwise, $\gcd (x - x_1, n)$ is either $p$ or $q$,
so Bob can now factor $n$.
Bob computes $y = s_b \oplus v$, where $v = 0$
if he knows the factorization of $n$ and $1$ otherwise.
Alice selects a large random string and sets the middle bit to $s_a$.
She encrypts this string using RSA and sends the result to Bob.

If Bob can factor $n$, then he knows Alice's private key
and simply decrypts Alice's message,
then uses $s_b$, revealing to Alice that he knows $s_a$.
Alice then knows that $y = s_b$.
The problem is that if Bob cannot factor $n$,
then Alice knows that $s_b = \overline y$, but Bob does not know $s_a$.
We can fix this problem by having Alice and Bob do the procedure above simultaneously.
This way, if at least one of them has the other's private keys,
then they both will know the secret bits by the reasoning above.
Otherwise, they will simply flip $y_a$ and $y_b$ to obtain the secrets.

{\section The Sander-Young-Yung System}

After Rivest, Adleman, and Dertouzos conjectured about the existence
of fully homomorphic encryption schemes than are secure and
allow the computation of encryptions of sums and products in polynomial time,
it was an open question until Gentry's work in 2009.
During this time, several groups made progress toward a solution.
In 1991, Tomas Sander, Adam Young, and Moti Yung produced a solution
for logarithmic depth circuits.
In their model, Alice has an input $x$ and Bob has a circuit $C$.
Let $E$ be an encryption function.
We want Bob to compute $E(C(x))$ without giving Bob information about $x$
or Alice information about $C$.

The solution revolves around ``inattentive evaluation if circuits,''
in which computation is performed relies on an additive homomorphism
and does not require knowledge of the plaintext values of the inputs.
We do this by representing the output of a bit operation as a tuple of four bits.
For example, $x \vee y$ is represented as $(x, y, x + y, 1)$.
Notice that if $x \equiv 1 \pmod 2$ or $y \equiv 1 \pmod 2$,
then the resulting tuple has three $1$'s and one $0$.
If $x \equiv y \equiv 0 \pmod 2$, then the resulting tuple
has three $0$'s and one $1$.
The OR gate can then return a random permutation if this tuple.
Alice then recovers the result by decrypting each element of the tuple
and seeing whether $0$ or $1$ appears three times.
We have therefore made progress toward extending additive homomorphism
to homomorphism with respect to bitwise OR.
Sander et al.~further extend this concept to other operations.

{\section The Boneh-Goh-Nissim Cryptosystem}

In 2006, Dan Boneh, Eu-Jin Goh, and Kobbi Nissim
give a system that bears some resemblance to the Paillier cryptosystem
and allows homomorphic encryption of 2-DNF formulas,
giving homomorphism under arbitrary additions and up to one multiplication.

The security of their scheme depends on the hardness of the subgroup decision problem.
Let $q_1$ and $q_2$ be random primes.
Let $G$ and $G_1$ be two cyclic groups of order $n = q_1 q_2$.
Let $g$ be a generator of $G$.
Let $e : G \times G \to G_1$ be a bilinear function, that is,
for any integers $a$ and $b$ and any elements $u$ and $v$ of $G$,
$e(u^a, v^b) = e(u, v)^{ab}$.
Additionally, $e$ must be such that $e(g,g)$ is a generator of $G_1$.
Given $n$, $G$, $G_1$, $e$, and an element $x$ of $G$,
is there a subgroup of $G$ containing $x$?

In order to create a key pair,
Alice selects two random generators $g$ and $u$ of $G$
and takes $h = u^{q_2}$.
Notice that $h$ generates a subgroup of $G$ of order $q_1$.
Alice publishes $G$, $G_1$, $e$, $n$, $g$, and $h$ as the public key
and keeps $q_1$ as the private key.
We can encrypt a message $m \in \{0,1,\ldots,T\}$, where $T < q_2$.
The encryption function is $E_r : m \mapsto g^m h^r$,
where $r$ is chosen at random from $\integer _n$.
The decryption function is $D : c \mapsto c^{q_1}$.
This function helps to recover $m$ because
$c^{q_1} = (g^m h^r)^{q_1} = g^{q_1 m} h^{q_1 r} = g^{q_1 m} 1^r = g^{q_1 m}$.
Completing the decryption requires finding
the discrete logarithm of $g^{q_1 m}$ base $g^{q_1}$.
This can be done in $O(\sqrt T)$ time.
Notice that we can only encrypt short messages with this system.

Since
$$\eqalign{
E_{r_1} (m_1) E_{r_2} (m_2) &= g^{m_1} h^{r_1} g^{m_2} h^{r_2}\cr
&= g^{m_1 + m_2} h^{r_1 + r_2}\cr
&= E_{r_1 + r_2} (m_1 + m_2)\cr
}$$
the system is additively homomorphic.
Notice that this property holds in $G_1$ as well.

To allow one multiplication,
take $g_1 = e(g,g)$ and $h_1 = e(g,h)$.
Since $e$ is bilinear, we have that $g_1$ generates $G_1$
and $h_1$ has order $q_1$.
Notice that since $g$ is a generator of $G$, there is some integer $\alpha$
such that $g^\alpha = u$.
Therefore, $h = u^{q_2} = g^{\alpha q^2}$.
In order to obtain an encryption of $m_1 m_2$ given encryptions $c_1$ and $c_2$
of $m_1$ and $m_2$, respectively, take a random $r \in \integer _n$ and set
$$\eqalign{
c &= e(c_1, c_2) h_1 ^r\cr
&= e(g^{m_1} h^{r_1}, g^{m_2} h^{r_2}) h_1 ^r\cr
&= e(g^{m_1} g^{\alpha q_2 r_1}, g^{m_2} g^{\alpha q_2 r_2}) h_1 ^r\cr
&= e(g^{m_1 + \alpha q_2 r_1}, g^{m_2 + \alpha q_2 r_2}), h_1 ^r\cr
&= e(g,g)^{(m_1 + \alpha q_2 r_1)(m_2 + \alpha q_2 r_2)} h_1 ^r\cr
&= g_1^{m_1 m_2 + m_1 \alpha q_2 r_2 + m_2 \alpha q_2 r_1 + \alpha ^2 q_2 ^2 r_1 r_2} h_1 ^r\cr
&= g_1^{m_1 m_2} g_1^{m_1 \alpha q_2 r_2} g_1^{m_2 \alpha q_2 r_1} g_1^{\alpha ^2 q_2^2 r_1 r_2} h_1 ^r\cr
&= g_1^{m_1 m_2} h_1^{m_1 r_2} h_1^{m_2 r_1} h_1^{\alpha q_2 r_1 r_2} h_1 ^r\cr
&= g_1^{m_1 m_2} h_1^{m_1 r_2 + m_2 r_1 + \alpha q_2 r_1 r_2 + r}\cr
}$$

{\section The Ishai-Paskin Cryptosystem}

In 2007, Yuval Ishai and Anat Paskin constructed a homomorphic
encryption scheme that allows evaluation of branching programs.
In this solution, they develop a variant of oblivious transfer
called ``strong oblivious transfer.''

{\section Gentry's Cryptosystem}

In 2009, Craig Gentry creates the first secure,
fully homomorphic encryption scheme that allows
computation of the encryption of arbitrary sums and products
in polynomial time.
A major innovation in his paper
is his quantification of ``noise'' in ciphertexts,
which grows as more operations are performed on them.
He gives a ``bootstrapping operation'' that reduces noise,
allowing a scheme that supports a limited number of operations
to support an arbitrary number of operations.
The security of the original version of his scheme
relies on the hardness of problems related to ideal lattices.

{\section Improvements and Implementations for Gentry's Cryptosystem}

While Gentry's solution is polynomial-time computable,
the large constant factor inhibited practical use at first.
When Gentry and Shai Halevi implemented his scheme in 2010,
they found that a single bootstrapping operation
could take up to thirty minutes, depending on the size of the public key.

In addition, there have also been schemes proposed that build on the work
of Gentry's 2009 paper.
Shai Halevi and Victor Shoup continue to work on an implementation
of one such scheme, by Zvika Brakerski, Craig Gentry, and Vinod Vaikuntanathan.

\bigskip
\centerline{Works Cited}
\everypar{\hangindent=20pt}
\parindent=0pt

Dan Boneh, Eu-Jin Goh, and Kobbi Nissim.
{\sl Evaluating 2-DNF Formulas on Ciphertexts}
Theory of Cryptography Conference.
2006.

Zvika Brakerski, Craig Gentry, and Vinod Vaikuntanathan.
{\sl Fully Homomorphic Encryption without Bootstrapping.}
Innovations in Theoretical Computer Science.
2011.

Taher ElGamal.
{\sl A Public Key Cryptosystem and a Signature Scheme Based on Discrete Logarithms.}
IEEE Transactions on Information Theory, Vol. IT-31, No. 4.
July 1985.

Craig Gentry.
{\sl A Fully Homomorphic Encryption Scheme (Ph.D. Thesis).}
2009.

Craig Gentry and Shai Halevi.
{\sl Implementing Gentry's Fully Homomorphic Encryption Scheme.}
EUROCRYPT.
2011.

Shai Halevi and Victor Shoup.
{\sl HElib: an Implementation of homomorphic encryption.}
https://github.com/shaih/HElib.
2018.

Yuval Ishai and Anat Paskin.
{\sl Evaluating Branching Programs on Encrypted Data.}
Theory of Cryptography Conference.
2007.

Hsiao-Ying Lin and Wen-Guey Tzeng.
{\sl An Efficient Solution to The Millionaires' Problem Based on Homomorphic Encryption.}
ACNS 2005.

Pascal Paillier.
{\sl Public-Key Cryptosystems Based on Composite Degree Residuosity Classes.}
Advances in Cryptography --- EUROCRYPT '99. pages 223--238.
EUROCRYPT 1999.
Lecture Notes in Computer Science, Vol. 1952. Springer, Berlin, Heidelberg.

Michael O. Rabin.
{\sl How to exchange secrets by oblivious transfer.}
Technical Report TR-81, Aiken Computation Laboratory, Harvard University.
1981.

Ronald L. Rivest, Len Adleman, and Michael L. Dertouzos.
{\sl On data banks and privacy homomorphisms.}
Foundations of Secure Computation.
1978.

Tomas Sander, Adam Young, and Moti Yung.
{\sl Non-Interactive Crypto Computing for {\it NC}${}^1$.}
32nd Annual IEEE Conference on Foundations of Computer Science.
October 1991.

Andrew Chi-chih Yao.
{\sl How to generate and exchange secrets.}
27th Annual IEEE Symposium on Foundations of Computer Science.
1986.

Danny the substitute lecturer.
Cryptography (Math 348) course at Rutgers University.
Spring 2018.
\bye
